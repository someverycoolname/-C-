\section{Сериализация и десериализация, работа с форматом JSON/XML}

Сериализация — процесс преобразования состояния объекта из обычного его представления, в вид пригодный для его перемещения и хранения между вызовами программы.

\subsection{JSON сериализация} \label{jsonxmlserialization}

JSON (JavaScript Object Notation) — формат распространения информации, пользующийся структурой "ключ-значение". Пример JSON объекта можно найти здесь:

\url{http://date.jsontest.com}

Данный сайт на запрос возвращает JSON, в котором хранится текущее время. Как пример такого ответа:

\begin{minted}{json}
{
   "time": "11:39:00 PM",
   "milliseconds_since_epoch": 1531352340670,
   "date": "07-11-2018"
}
\end{minted}

где \mintinline{json}{"time"} и \mintinline{json}{"date"} записаны время и дата соответственно в более знакомом для людей формате \mintinline{csharp}{hh:mm:ss PM/AM mm-dd-yyyy}, а \mintinline{json}{"milliseconds_since_epoch"} — UNIX timestamp, количество миллисекунд с начала эпохи, то есть с 00:00:00 UTC 1 января 1970 года.  

Создадим класс, в котором будут хранится данные о результатах игрока. Нам важно знать его имя, количество им полученных очков и сколько времени он провел в игре. Для того чтобы иметь возможность сериализовать объект мы должны пометить, что этот объект сериализируемый. В частности, мы хотим объявить о том, что объект класса возможно сериализовать с помощью \mintinline{csharp}{DataContractSerializer}. 

Для этого нам потребуется подключить пространство имен

\begin{minted}{csharp}
using System.Runtime.Serialization;
\end{minted}

и реализовать класс \mintinline{csharp}{PlayerResult}

\begin{minted}{csharp}
[DataContract]
class PlayerResult 
{
    [DataMember] 
    public string PlayerName { get; set; }

    [DataMember] 
    public int Score { get; set; }

    [DataMember] 
    public TimeSpan TimeDuration { get; set; }

    public override string ToString()
    {
        return PlayerName + " " + Score + " " + TimeDuration;
    }
}
\end{minted}

Осталось только сериализовать и десериализовать объект. Для удобства, в данном примере мы будем использовать \mintinline{csharp}{MemoryStream}. Для сериализации объекта будет использован класс \mintinline{csharp}{DataContractJsonSerializer}, который содержится в пространстве имен

\begin{minted}{csharp}
using System.Runtime.Serialization.Json
\end{minted}

Теперь напишем это в коде нашей программы

\begin{minted}{csharp}
var memoryStream = new MemoryStream();
var jsonSerializer = new DataContractJsonSerializer(typeof(PlayerResult));
\end{minted}

Сериализация объекта происходит с помощью метода \mintinline{csharp}{WriteObject(Stream, Object)}. Для начала объявим объект (называться он будет \mintinline{csharp}{pr}), а после сериализируем его

\begin{minted}{csharp}
var pr = new PlayerResult();
pr.PlayerName = "Alice";
pr.Score = 42;
pr.TimeDuration = new TimeSpan(0, 12, 32);

jsonSerializer.WriteObject(memoryStream, pr);
\end{minted}

Чтобы посмотреть что получилось, давайте воспользуемся объектом \mintinline{csharp}{StreamReader(Stream)}

\begin{minted}{csharp}
var streamReader = new StreamReader(memoryStream);
memoryStream.Position = 0;
Console.WriteLine(streamReader.ReadToEnd());
\end{minted}

Как результат исполнения вы можете получить вот такой вывод

\begin{minted}{json}
{"PlayerName":"Alice","Score":12,"TimeDuration":"PT12M32S"}
\end{minted}


Не забудьте освободить ресурсы, выделенные для MemoryStream.
\begin{minted}{csharp}
memoryStream.Dispose();
\end{minted}

Теперь десериализуем этот объект. Для этого воспользуемся методом \mintinline{csharp}{ReadObject(Stream)}

\begin{minted}{csharp}
memoryStream.Position = 0;
var pr_copy = (PlayerResult)jsonSerializer.ReadObject(memoryStream);
Console.WriteLine(pr_copy);
\end{minted}

В результате исполнения этого участка кода мы получим

\begin{minted}{csharp}
Alice 12 00:12:32
\end{minted}

Более сложный пример с использованием сериализации был рассмотрен в \ref{webrequest}. В нем используется контейнер \mintinline{csharp}{List<>} и если есть вопросы относительно этого, загляните туда.

\subsection{XML сериализация}

Примером типичного XML файла может полужить любой \path{*.xaml} файл, с которым вы скорее всего работали ранее по ходу этих методических материалов. Также примером может послужить ответ сервера PLOS (Public Library of Science, Открытая научная библиотека) на запрос

\url{http://api.plos.org/search?q=title:"Graph"}

который запрашивает информацию о всех статьях со словом Graph (граф) в названии.  

Для того чтобы сериализовать и десериализовать объект в XML разметке нужно использовать класс \mintinline{csharp}{DataContractSerializer}, который находится  в пространстве имен

\begin{minted}{csharp}
using System.Runtime.Serialization
\end{minted}

Он работает индентично уже рассмотренному \mintinline{csharp}{DataContractJsonSerializer}. Более того, оба класса являются наследниками класса \mintinline{csharp}{XmlObjectSerializer}. Чтобы увидеть как это работает, достаточно заменить

\begin{minted}{csharp}
var jsonSerializer = new DataContractJsonSerializer(typeof(PlayerResults));
\end{minted}

на 

\begin{minted}{csharp}
var jsonSerializer = new DataContractSerializer(typeof(PlayerResults));
\end{minted}

Оставим название класса, для уменьшения количества правок в коде, но не забывайте о функции рефакторинга в вашей IDE. 

Как результат наших манипуляций, наш объект \mintinline{csharp}{pr} после сериализации выглядит так

\begin{minted}{csharp}
<PlayerResults 
	xmlns="http://schemas.datacontract.org/2004/07/serialization_test"
	xmlns:i="http://www.w3.org/2001/XMLSchema-instance">

		<PlayerName>Alice</PlayerName>
		<Score>12</Score>
		<TimeDuration>PT12M32S</TimeDuration>
		
</PlayerResults>
\end{minted}

Освободим ресурсы, выделенные для MemoryStream.
\begin{minted}{csharp}
memoryStream.Dispose();
\end{minted}

% \subsection{Использование атрибута \mintinline{csharp}{Serializable}}

% \subsection{Использование атрибута \mintinline{csharp}{DataContract}}
