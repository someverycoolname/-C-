\section{Класс \mintinline{csharp}{WebRequest} и его производные. Работа с Web API} 
\subsection{Использование производных класса WebRequest}
Класс \mintinline{csharp}{WebRequest} позволяет делать запоросы на веб-сервер. Этот класс является абстрактным и от него наследуются различные классы, которые отличаются друг от друга протоколом. В частности это: 

\begin{description}[style=nextline]
\item [\mintinline{csharp}{System.Net.HttpWebRequest}] использует протокол HTTP (HyperText Transfer Protocol, Протокол Передачи Гипертекста)
\item [\mintinline{csharp}{System.Net.FtpWebRequest}]
ипользует протокол FTP (File Transfer Protocol, Протокол Передачи Данных)
\item [\mintinline{csharp}{System.Net.FileWebRequest}]
ипользует протокол \path{file}
\item [\mintinline{csharp}{System.IO.Packaging.PackWebRequest}] в отличие от предыдущих базируется на идее отправки запросов находящихся в пакетах. В частности в ZIP.
\end{description}

\subsection{Пример} \label{webrequest} 

В \ref{jsonxmlserialization} было рассказано про JSON сериализацию. Давайте применим эти навыки в более приближенном к реальности примере. На просторах сети Интернет можно найти множество ресурсов с тестовыми JSON документами. Воспользуемся одним из них

\url{https://jsonplaceholder.typicode.com/users}

Здесь содержится какая база данных выдуманных пользователей. Наша задача получить её и десериализировать. На выходе у нас получится \mintinline{csharp}{List<>} с пользователями.

\textit{Многие места в примерах кода были сокращены, поскольку занимают слишком много места в тексте.}

Для начала мы должны рассмотреть запрашиваемый документ и подстроить под него архитектуру наших будующих классов. 

\begin{minted}{json}
{
    "id": 1,
    "name": "Leanne Graham",
    "username": "Bret",
    "email": "Sincere@april.biz",
    "address": { ... },
    "phone": "1-770-736-8031 x56442",
    "website": "hildegard.org",
    "company": { ... }
}
\end{minted}

\newpage

Наша задача описать классы для каждой структуры: \mintinline{json}{"address"}, \mintinline{json}{"company"} и для структуры, которая содержит все эти поля, назовем её  \mintinline{json}{"user"}. Для этого можно воспользоваться ресурсом, который автоматически создаст архитектуру классов под нужный нам JSON

\url{http://json2csharp.com/}

Единственное что нам останется сделать, это описать каждый класс и каждое поле как сериализируемые. Как напоминание, это делается с помощью атрибутов \mintinline{csharp}{[DataContract]} и \mintinline{csharp}{[DataMember]}. Как итог у нас должно получиться что-то подобное:

\begin{minted}{csharp}
[DataContract]
public class Geo { ... }

[DataContract]
public class Address { ... }

[DataContract]
public class Company { ... }

[DataContract]
public class User
}
    [DataMember]
    public int id { get; set; }

    [DataMember]
    public string name { get; set; }

    [DataMember]
    public string username { get; set; }

    [DataMember]
    public string email { get; set; }

    [DataMember]
    public Address address { get; set; }

    [DataMember]
    public string phone { get; set; }

    [DataMember]
    public string website { get; set; }

    [DataMember]
    public Company company { get; set; }

    public override string ToString()
    {
        return id + " " + name + " " + username + " " + phone;
    }
}
\end{minted}

Далее, мы должны сформировать запрос. Для этого давайте воспользуемся статическим методом \mintinline{csharp}{WebRequest.Create(String)}

\begin{minted}{csharp}
WebRequest request = WebRequest.Create("https://jsonplaceholder.typicode.com/users");
\end{minted}

Теперь мы хотим отправить наш запрос серверу и получить на него ответ

\begin{minted}{csharp}
HttpWebResponse response = (HttpWebResponse)request.GetResponse();
\end{minted}

Класс \mintinline{csharp}{HttpWebResponse} является простой оберткой вокруг ответа HTTP сервера. Нас от него интересует в данный момент всего две вещи: 

\begin{enumerate}
    \item Удостовериться, что запрос был принят и мы получили ответ от сервера
    \item Получить ответ
\end{enumerate}

Выведем статус. Далее воспользуемся потоками для чтения из ответа

\begin{minted}{csharp}
Console.WriteLine(response.StatusDescription);

Stream dataStream = response.GetResponseStream();
StreamReader reader = new StreamReader(dataStream);
\end{minted}

Далее уже все очень просто и знакомо

\begin{minted}{csharp}
var jsonSerializer = new DataContractJsonSerializer(typeof(List<User>));
var users = (List<User>)jsonSerializer.ReadObject(dataStream);
\end{minted}

Выведем на экран

\begin{minted}{csharp}
foreach (var it in users) {
    Console.WriteLine(it);
}
\end{minted}

Как итог, мы должны получить что-то такое

\begin{minted}{csharp}
OK
1 Leanne Graham Bret 1-770-736-8031 x56442
2 Ervin Howell Antonette 010-692-6593 x09125
3 Clementine Bauch Samantha 1-463-123-4447
4 Patricia Lebsack Karianne 493-170-9623 x156
5 Chelsey Dietrich Kamren (254)954-1289
6 Mrs. Dennis Schulist Leopoldo_Corkery 1-477-935-8478 x6430
7 Kurtis Weissnat Elwyn.Skiles 210.067.6132
8 Nicholas Runolfsdottir V Maxime_Nienow 586.493.6943 x140
9 Glenna Reichert Delphine (775)976-6794 x41206
10 Clementina DuBuque Moriah.Stanton 024-648-3804
\end{minted}

\textit{Не забывайте закрывать все открытые потоки, либо используйте \mintinline{csharp}{using}}
